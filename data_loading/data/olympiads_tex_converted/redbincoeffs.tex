\documentclass[11pt]{article}
\usepackage{latexsym, amsthm, amsfonts, amsmath}
\usepackage{amssymb}
\setlength{\textwidth}{6.5in} \setlength{\oddsidemargin}{0in}
\setlength{\textheight}{8.5in} \setlength{\topmargin}{0in}
\setlength{\headheight}{0in} \setlength{\headsep}{0in}
\setlength{\parskip}{0pt} \setlength{\parindent}{20pt}

\def\bi{\begin{itemize}}
\def\ei{\end{itemize}}
\def\ii{\item}

\def\be{\begin{enumerate}}
\def\ee{\end{enumerate}}
\def\beqa{\begin{eqnarray}}
\def\eeqa{\end{eqnarray}}

\def\ol{\overline}
\def^{\text th}{^{\mathrm {th}}}
\def\Rc{\rceil}
\def\Lc{\lceil}
\def\Rf{\rfloor}
\def\Lf{\lfloor}
\def\Lp{\left(}
\def\Rp{\right)}

\begin{document}

{\bf MOSP 2005, Red}

{\bf Binomial Coefficients}

\bigskip

{\bf Definition} {\it The binomial coefficient $\binom{n}{k}$ is
defined to be} $$\binom{n}{k} = \frac{n!}{k!(n-k)!}.$$

The binomial coefficient $\binom{n}{k}$ can be understood as the
number of subsets of size $k$ of an $n$-element set (in other
words, the number of ways to choose $k$ objects from a collection
of $n$ objects).  By way of convention, if ($k$ is an integer and)
$k<0$ or $k>n$, $\binom{n}{k} = 0$.  Since the formula for the
binomial coefficient can be written as $$\binom{n}{k} =
\frac{n(n-1)(n-2)\cdots(n-k+1)}{k!},$$ we may think of the
binomial coefficient as a polynomial in $n$, in fact defined for
all $x$.  The convention for $k>n$ as stated above is consistent
with this.\\

{\bf Theorem 1 (The Binomial Theorem).} The following formula
holds:
$$(x+y)^n = \binom{n}{0}x^n + \binom{n}{1}x^{n-1}y + \binom{n}{2}
x^{n-2}y^2 +\cdots + \binom{n}{n-1}xy^{n-1}+ \binom{n}{n} y^n.$$\\

{\it Proof.} Let us determine the coefficient of $x^ky^{n-k}$.  If
we write out the product as $$(x+y)(x+y)(\cdots)(x+y),$$ we can
observe that for each way of choosing $k$ of these multiplicands,
there is a contribution of 1 to the coefficient of $x^ky^{n-k}$,
since we can think of $x$'s from the chosen multiplicands being
multiplied by the $y$'s from those not chosen.  Each contribution
to the coefficient in question arises uniquely in this way, and so
we have identified the two quantities, as we wanted.\\

{\bf Theorem 2.} {\it Let $n$ and $k$ be positive integers with $n
\ge k$. The following fundamental properties of the binomial
coefficients $\binom{n}{k}$ hold:} \be \ii [(a)] $\binom{n}{k} =
\binom{n}{n-k}$; \ii [(b)] $\binom{n}{k+1} = \binom{n-1}{k+1} +
\binom{n-1}{k}$; \ii [(c)] $\binom{n}{0} < \binom{n}{1} <
\binom{n}{2} < \cdots < \binom{n}{\Lc\frac{n-1}{2}\Rc} =
\binom{n}{\Lf\frac{n}{2}\Rf}$; \ii [(d)] $k\binom{n}{k} =
n\binom{n-1}{k-1}$; \ii [(e)] $k\binom{n}{k} =
(n-k+1)\binom{n}{k-1}$; \ii [(f)] $\binom{n}{0} + \binom{n+1}{1} +
\binom{n+2}{2} + \cdots + \binom{n+k}{k} = \binom{n+k+1}{k}$; \ii
[(g)] $\binom{n}{n} + \binom{n+1}{n} + \binom{n+2}{n} + \cdots +
\binom{n+k}{n} = \binom{n+k+1}{n+1}$; \ii [(h)] $\binom{n}{0} +
\binom{n}{1} + \cdots + \binom{n}{n} = 2^n$; \ii [(i)]
$\binom{n}{0} - \binom{n}{1} + \binom{n}{2} - \cdots +
(-1)^n\binom{n}{n} = 0$; \ii [(j)] $\binom{n}{1} + 2\binom{n}{2} +
3\binom{n}{3} + \cdots + n\binom{n}{n} = n2^{n-1}$; \ii [(k)]
$\binom{n}{k}$ is divisible by $n$ if $n$ is prime and $1\le k \le
n-1$. \ee

{\it Proof.} Routine, discussed in lecture.\\

Binomial coefficients can be viewed as entries of Pascal's
triangle; the entry in the $n^{\text th}$ row and $k^{\text th}$ column (where
row and column indexing starts at 0) is $\binom{n}{k}$.

Now for some examples.

{\bf Example 1.}\quad {[AIME 1989]} Ten points are marked on a
circle. How many distinct convex polygons of three or more sides
can be drawn using some (or all) of the ten points as vertices?
(Polygons are distinct unless they have exactly the same
vertices.)\\

{\bf Example 2.}\quad Evaluate
\[
\frac{\binom {11}{0}}{1} + \frac{\binom {11}{1}}{2} + \frac{\binom
{11}{2}}{3} + \cdots + \frac{\binom{11}{11}}{12}.
\]\\

{\bf Example 3.}\quad Let $n$ be a positive integer. Prove that
\[
\sum_{k=1}^{n} \frac{(-1)^{k-1}}{k}\binom{n}{k} = 1 + \frac{1}{2}
+ \cdots + \frac{1}{n}.
\]\\

{\bf Example 4.}\quad Let $\{ F_n\}_{n=0}^{\infty}$ be the
sequence defined by $F_0 = F_1 = 1$ and $F_{n+2} = F_{n+1} +
F_{n}$ for $n \ge 0$. Prove that
\[
\sum_{k=0}^{n}\binom{n-k+1}{k} = F_{n+1}.
\]\\

{\bf Example 5.}\quad
%xinaoshu 11
{[Vandermonde]} Let $m, n$, and $k$ be integers with $m ,n \ge 0$.
Prove that
\[
\binom{m+n}{k} = \sum_{i=0}^{k} \binom{m}{i}\binom{n}{k-i}.
\]\\

{\bf Theorem 3.}\quad {\it {[Lucas]} Let $p$ be a prime, and let
$n$ be a positive integer with $n = (\ol{n_mn_{m-1}\dots
n_{0}})_p$. Let $i$ be a positive integer less than n. Also, write
$i = i_0 + i_1 p + \cdots + i_m p^m$, where $0 \le i_0, i_1,
\dots, i_m \le p-1$. Then}
$$
\binom{n}{i} \equiv \prod_{j=0}^{m} \binom{n_j}{i_j} \pmod{p}.
\eqno{(**)}
$$
{\it Here $\binom{0}{0} = 1$ and $\binom{n_j}{i_j} = 0$ if $n_j <
i_j$ by the convention we set earlier.}\\

{\it Proof.} One good argument (which you should be able to work
out on your own) involves the identity
$$(x+y)^p\equiv x^p + y^p \pmod{p}.$$\\

One useful generalization of binomial coefficients are multinomial
coefficients.  They are defined as $$\binom{n}{n_1, n_2, \dots ,
n_m} = \frac{n!}{n_1!n_2!\cdots n_m!},$$ where $n_1+\cdots+n_m =
n.$  The above multinomial coefficient counts the number of ways
of partitioning an $n$-element set into parts of size
$n_1,n_2,\ldots,n_m$.  As before, the convention is that if any of
the (integral) $n_k$ are negative then the multinomial coefficient
is $0$.  The expected generalization of Theorem 1 above indeed
works:\\

{\bf Theorem 4.}\quad {\it Let $m$ and $n$ be positive integers.
Then} \beqa
& & (x_1+x_2+\cdots + x_m)^n \\
& = & \sum_{\stackrel{\scriptstyle n_1, n_2, \dots , n_m \ge 0}
{n_1+n_2+\cdots+n_m = n}} \binom{n}{n_1, n_2, \dots , n_m}
x_1^{n_1}x_2^{n_2}\cdots x_m^{n_m}. \eeqa\\

{\it Proof.} Similar to the Binomial Theorem.\\

\begin{center}
{\bf Exercises and Problems}
\end{center}

\bigskip

\be \ii [1.] Let $n, m, k$ be nonnegative integers such that $m\le
n$. Show that
\[
\binom{n}{k} \binom{k}{m} = \binom{n}{m}\binom{n-m}{k-m}.
\]

\ii [2.] What is the value of the constant term in the expansion
of $\Lp{x^2+ \frac{1}{x^2}-2}\Rp^{10}$?

\ii [3.] Let $n$ be a positive integer. Prove that
\[
\sum_{k=0}^{n} k^2\binom{n}{k} = n(n+1)2^{n-2}\quad \mbox{and}
\quad \sum_{k=0}^{n} \frac{(-1)^k}{k+1} \binom{n}{k} =
\frac{1}{n+1}.
\]

\ii [4.]
%xinaoshu 14
Let $n$ be a nonnegative integer. Prove that
\[
\sum_{i = 0}^k (-1)^i\binom{n}{i} = (-1)^k \binom{n-1}{k}.
\]

\ii [5.] %typed by tony
Let $n$ be an odd positive integer. Prove that the array
\[
\binom{n}{0},\ \binom{n}{1},\ \dots,\ \binom{n}{\frac{n-1}{2}}
\]
contains an even number of odd numbers.

\ii [6.] %Hua3/136
Let $n$ be a positive integer. Determine the number of odd numbers
in $\binom{n}{0}, \binom{n}{1}, \dots , \binom{n}{n}$.

\ii [7.] %typed by tony
Prove that the binomial coefficients $\binom{2^n}{k}$, $k=1 , 2,
\dots, 2^{n-1}-1, 2^{n-1}+1, \dots, 2^n-1$, are all divisible by
4.

\ii [8.] Let $p$ be any prime number. Prove that
\[
\binom{2p}{p} \equiv 2 \pmod{p^2}.
\]


\ii [9.]
%xinaoshu 3/15
Let $n$ be a nonnegative integer. Show that
\[
\sum_{k=0}^{n} k\binom{n}{k}^2 = n\binom{2n-1}{n-1}.
\]



\ii [10.]
%Med
[MOSP 2001, Cecil Rousseau] Let $a_n$ denote the number of
nonempty sets $S$ such that \be \ii [(i)] $S\subseteq \{ 1, 2,
\dots , n\}$; \ii [(ii)] all elements of $S$ have the same parity;
\ii [(iii)] each element $k \in S$ satisfies $k \ge 2|S|$, where
$|S|$ denotes the number of elements in $S$. \ee Prove that
\[
a_{2m-1} = 2(F_{m+1} - 1)\quad {\mbox{and}}\quad a_{2m} = F_{m+3}
- 2
\]
for all $m \ge 1$, where $F_n$ is the $n^{\text th}$ Fibonacci number.

\ii [11.] [IMO 1995/6] Let $p$ be an odd prime, and let $S$ denote
the set of positive integers from $1$ to $2p$.  How many subsets
of $S$ of size $p$ have a sum divisible by $p$?

\ii [12.] [TST 2000, from Kvant] Let $n$ be a positive integer.
Prove that
\[
\sum_{i=0}^{n}\binom{n}{i}^{-1} = \frac{n+1}{2^{n+1}}
\sum_{i=1}^{n+1} \frac{2^{i}}{i}.
\]

\ii [13.] Let $p>3$ be a prime.  Prove that $$\binom{2p}{p}\equiv
2\pmod{p^3}.$$

\ee

\end{document}
