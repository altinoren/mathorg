\documentclass[11pt]{article} 
\usepackage{amsfonts, amsmath, amssymb, amsthm, fullpage, mdwlist, fancyhdr} 
 
\setlength{\textheight}{9.25in} 
\setlength{\textwidth}{6.5in} 
\setlength{\topmargin}{0.0in} 
\setlength{\headheight}{0.0in} 
\setlength{\headsep}{0.0in} 
\setlength{\leftmargin}{0.0in} 
\setlength{\oddsidemargin}{0.0in} 
\setlength{\parindent}{0pc} 
 
 
\begin{document} 

\title{Smoothing}
\author{Alex Zhu}
\date{KSA Math Olympiad Training 2013}

\maketitle

\section{Useful Facts}
\begin{enumerate}
  \item \textbf{Jensen's inequality:} Suppose $f$ is a function convex over some interval $I$. Then if $x_1, x_2, \ldots, x_n$ lie in $I$, then \[ \frac{f(x_1) + f(x_2) + \cdots + f(x_n)}{n} \geq f \left( \frac{x_1 + x_2 + \cdots + x_n}{n} \right). \] In other words, if $f$ is convex and $\sum x_i$ is fixed, $\sum f(x_i)$ is minimized when all $x_i$'s are equal. 
  \item \textbf{Karamata's inequality:} Suppose $f$ is a convex function over some interval $I$. If $x_1 \geq x_2 \geq \cdots \geq x_n$ and $y_1 \geq y_2 \geq \cdots \geq y_n$ are reals in $I$ such that the sequence $(x_1, x_2, \ldots, x_n)$ majorizes $(y_1, y_2, \ldots, y_n)$, then \[ f(x_1) + f(x_2) + \cdots + f(x_n) \geq f(y_1) + f(y_2) + \cdots + f(y_n). \] 
  (The sequence $(x_1, x_2, \ldots, x_n)$ is said to \emph{majorize} $(y_1, y_2, \ldots, y_n)$ if $x_1 + x_2 + \cdots + x_n = y_1 + y_2 + \cdots + y_n$, and $x_1 + x_2 + \cdots + x_k \geq y_1 + y_2 + \cdots + y_k$ for $1 \leq k \leq n-1$.)
  \item \textbf{Convex maxima:} If $f$ is a convex function over an interval $I$, then the maximum of $f$ is attained at an endpoint of $I$. 
\end{enumerate}

\section{Common Techniques}
\begin{itemize}
  \item \textbf{Tangent line approximation:} If we are trying to minimize the sum $f(x_1) + f(x_2) + \cdots + f(x_n)$ given that $x_1 + x_2 + \cdots + x_n = S$, it can be helpful to compare $f$ with the linear function $ax + b$ that is tangent to $f$ at $\frac{S}{n}$.  
  \item \textbf{Ad hoc comparisons:} Suppose $f$ is a function. The crux of smoothing is that pushing variables $x$ and $y$ together or apart can guarantee $f(x) + f(y)$ to increase or decrease. Knowing whether $f$ is convex can significantly facilitate smoothing, but analyzing the change of $f(x) + f(y)$ is possible even when, say, $x$ lies in a concave region of $f$ while $y$ lies in a convex region. In such a case, an ad hoc algebraic comparison could be used instead. 
  \item \textbf{Introducing a constraint:} Sometimes, adding a constraint to a homogeneous inequality makes it equivalent to minimizing $\sum f(x_i)$ for some function $f$, making it susceptible to the smoothing methods. 
  \item \textbf{Variable substitution:} Smoothing works best when the variables have a fixed sum. If the variables satisfy some other sort of constraint, it can be helpful to perform a variable substitution such that the new variables have a fixed sum. For example, if we have variables $a,b,c$ subject to $abc = 1$, then if we perform the substitution $a = e^x, b = e^y, c = e^z$, our new constraint becomes $x+y+z=0$. 
\end{itemize}

\section{Problems}

\begin{enumerate}
  \item (Nesbitt) Show that if $a,b$, and $c$ are positive real numbers, then \[ \frac{a}{b+c} + \frac{b}{a+c} + \frac{c}{a+b} \geq \frac{3}{2}. \]
  
  \item If $x,y$, and $z$ are positive real numbers satisfying $x + y + z = xyz$, show that \[ \sqrt{1+x^2} + \sqrt{1+y^2} + \sqrt{1+z^2} \geq 6. \]

  \item (USAMO 1980) Prove that for real numbers $a,b,c$ in the interval $[0,1]$, \[ \frac{a}{b+c+1}+\frac{b}{c+a+1}+\frac{c}{a+b+1}+(1-a)(1-b)(1-c) \le 1. \]
  
  \item (USAMO 1977) If $ a,b,c,d,e$ are positive numbers bounded by $ p$ and $ q$, i.e, if they lie in $ [p,q], 0 < p$, prove that
\[ (a + b + c + d + e)\left(\frac {1}{a} + \frac {1}{b} + \frac {1}{c} + \frac {1}{d} + \frac {1}{e}\right) \le 25 + 6\left(\sqrt {\frac {p}{q}} - \sqrt {\frac {q}{p}}\right)^2\]
and determine when there is equality.
 
  \item (USAMO 1999) Let $a_{1}, a_{2}, \dots, a_{n}$ ($n > 3$) be real numbers such that \[ a_{1} + a_{2} + \cdots + a_{n} \geq n \qquad \mbox{and} \qquad a_{1}^{2} + a_{2}^{2} + \cdots + a_{n}^{2} \geq n^{2}.  \] Prove that $\max(a_{1}, a_{2}, \dots, a_{n}) \geq 2$. 
  
  \item (USAMO 2003) Let $ a$, $ b$, $ c$ be positive real numbers. Prove that
\[ \dfrac{(2a + b + c)^2}{2a^2 + (b + c)^2} + \dfrac{(2b + c + a)^2}{2b^2 + (c + a)^2} + \dfrac{(2c + a + b)^2}{2c^2 + (a + b)^2} \le 8. \] 

  \item (Poland 1996) Suppose $x,y,z$ are real numbers such that $x + y + z = 1$ and $x,y,z > -\frac{3}{4}$. Prove that \[ \frac{x}{1+x^2} + \frac{y}{1+y^2} + \frac{z}{1+z^2} \leq \frac{9}{10}. \] 
  
  \item (2009 ISL) Let $a,b,c$ be positive real numbers such that $\frac{1}{a} + \frac{1}{b} + \frac{1}{c} = a + b + c$. Prove that: \[ \frac{1}{(2a+b+c)^2} + \frac{1}{(a+2b+c)^2} + \frac{1}{(a+b+2c)^2} \leq \frac{3}{16}. \] 
  
  \item (Romania 99) Show that for all positive reals $x_1, x_2, \ldots, x_n$ with $x_1 x_2 \cdots x_n = 1$, we have \[ \frac{1}{n-1 + x_1} + \cdots + \frac{1}{n - 1 + x_n} \leq 1. \]
 
  \item (Japan 1997) Show that for all positive reals $a,b,c$, 
  \[ \frac{(a+b-c)^2}{(a+b)^2+c^2} + \frac{(b+c-a)^2}{(b+c)^2 + a^2} + \frac{(c+a-b)^2}{(c+a)^2 + b^2} \geq \frac{3}{5}. \]
  
  \item Show that if $r > 1$ and $x_1, x_2, \ldots, x_n$ are $n$ positive real numbers such that $x_1 x_2 \cdots x_n = 1$, then 
  \[ x_1^r + x_2^r + \cdots + x_n^r \geq x_1 + x_2 + \cdots + x_n. \]
    
  \item (USAMO 1998) Let $a_0,a_1,\cdots ,a_n$ be numbers from the interval $(0,\pi/2)$ such that \[ \tan (a_0-\frac{\pi}{4})+ \tan (a_1-\frac{\pi}{4})+\cdots +\tan (a_n-\frac{\pi}{4})\geq n-1.  \] Prove that \[ \tan a_0\tan a_1 \cdots \tan a_n \geq n^{n+1}.  \]
  
  \item The numbers $x_1, x_2, \ldots, x_n$ obey $-1 \leq x_1, x_2, \ldots, x_n \leq 1$ and $x_1^3 + x_2^3 + \cdots + x_n^3 = 0$. Prove that \[ x_1 + x_2 + \cdots + x_n \leq \frac{n}{3}. \]
   
  \item Let $\, a_1, a_2, a_3, \ldots \,$ be a sequence of positive real numbers satisfying $\, \sum_{j=1}^n a_j \geq \sqrt{n} \,$ for all $\, n \geq 1$. Prove that, for all $\, n \geq 1, \,$  \[ \sum_{j=1}^n a_j^2 > \frac{1}{4} \left( 1 + \frac{1}{2} + \cdots + \frac{1}{n} \right).  \]
  
  \item Let $a_1, a_2, \ldots, a_n$ be distinct positive integers such that each subset of $\{a_1, a_2, \ldots, a_n\}$ has a different sum. Prove that \[ \frac{1}{a_1} + \frac{1}{a_2} +  \cdots + \frac{1}{a_n} < 2. \]
  
  \item (USA TST 2007) Let $ n$ be a positive integer and let $ a_1 \leq a_2 \leq \cdots \leq a_n$ and $ b_1 \leq b_2 \leq \cdots \leq b_n$ be two nondecreasing sequences of real numbers such that
\[ a_1 + \cdots + a_i \leq b_1 + \cdots + b_i \text{ for every } i = 1,\ldots,n - 1 \]
and
\[ a_1 + \cdots + a_n = b_1 + \cdots + b_n. \]
Suppose that for any real number $ m$, the number of pairs $ (i,j)$ with $ a_i - a_j = m$ equals to the number of pairs $ (k,l)$ with $ b_k - b_l = m$. Prove that $ a_i = b_i$ for $ i = 1,\ldots,n$.

% \item (Erdos?) Define $\{u_i\}_{i\ge1}$ by $u_1=2$, $u_{i+1} = u_i^2 - u_i + 1$ for $i\ge1$. If $n$ is a positive integer and $a_1,a_2,\ldots,a_n$ is a sequence of numbers with $\frac{1}{a_1}+\cdots+\frac{1}{a_n}<1$, show that $\frac{1}{a_1}+\cdots+\frac{1}{a_n} \le \frac{1}{u_1}+\cdots+\frac{1}{u_n}$.

% \item (Romania TST) Let $z_1,z_2,\ldots,z_n$ be $n$ complex numbers in a closed disk $D$. Prove that there exists $z\in D$ satisfying $z^n = z_1z_2\cdots z_n$.

\end{enumerate}
\end{document}

1) smoothing operation
2) majorization 
  - proof for n=2,3
  - proof for general n
3) corollary: karamata
corollaries of those: 
  - weighted jensen
  - max/min of convex/concave are at endpoints
  - inflection point theorem