\documentclass[11pt]{article}
\usepackage{latexsym, amsthm, amsfonts, amsmath}
\usepackage{amssymb}
\setlength{\textwidth}{6.5in} \setlength{\oddsidemargin}{0in}
\setlength{\textheight}{8.5in} \setlength{\topmargin}{0in}
\setlength{\headheight}{0in} \setlength{\headsep}{0in}
\setlength{\parskip}{0pt} \setlength{\parindent}{20pt}

\def\bi{\begin{itemize}}
\def\ei{\end{itemize}}
\def\ii{\item}
\def\be{\begin{enumerate}}
\def\ee{\end{enumerate}}
\def\bd{\begin{description}}
\def\ed{\end{description}}
\def\beqa{\begin{eqnarray}}
\def\eeqa{\end{eqnarray}}
\def\msk{\medskip}

\def\ol{\overline}
\def^{\text th}{^{\mathrm {th}}}
\def\Rc{\rceil}
\def\Lc{\lceil}
\def\Rf{\rfloor}
\def\Lf{\lfloor}
\def\Lp{\left(}
\def\Rp{\right)}
\def\mathbb{N}{\mathbb N}

\newcounter{dfn}
\newcounter{theo}

\newtheorem{defn}[dfn]{Definition}
\newtheorem{lem}[theo]{Lemma}
\newtheorem{thm}[theo]{Theorem}
\newtheorem{cor}[theo]{Corollary}

\begin{document}

\noindent{\bf MOSP 2005, Red}

\noindent {\bf Generating Functions}

Suppose $(a_n)_{n=0}^\infty$ is a sequence of (complex) numbers.
We may form the following (formal) power series: \[F(x) =
\sum_{n=0}^\infty a_n x^n.\]  This is called the \emph{generating
function} of the sequence $(a_n)$.

Generating functions can be a very useful tool for moving between
combinatorics and algebra; they often stand in as closed-form
objects which represent the result of a recursion, even if that
recursion doesn't have a closed-form term-by-term solution.  One
value of generating functions is that they may be added,
subtracted, multiplied, and divided (under easy-to-check
hypotheses); one may raise a positive number to the power of a
generating function, or compose any generating function with
another (again, under suitable hypotheses).

One function which will come up a lot is the following geometric
series: \[a+arx+ar^2x^2+\cdots = \frac{a}{1-rx}.\]  The expression
on the right is how we will want to think of this generating
function (manipulations of this sort are what make generating
functions useful).  The way you may have learned to understand the
above equation, however, is in terms of convergent power series on
an interval of the real line.  This will \emph{not} turn out to be
very useful for us, at least initially.  In general we want to
look at sequences that might not give rise to actual functions
(like say $a_n = n^n$).  On the other hand, expressions which make
sense in the world of functions do not necessarily make sense in
the world of generating functions.  For example, consider
$f(x)=1/(1-x)$ as above and $g(x) = 2+x.$  The composition
$f(g(x))$ (considered purely generatingfunctionologically) would
take the form
\[1+(2+x)+(2+x)^2+(2+x)^3+\cdots = (1+2+2^2+\cdots)+(1+2\cdot 2+3\cdot 2^2+\cdots)x+
(1+3\cdot 2+6\cdot 2^2+\cdots)x^2 +\cdots,\] i.e. nonsense, even
though the composition function
\[\frac{1}{1-(2+x)}=-\frac{1}{1+x} = -1+x-x^2+x^3-\cdots\] is
perfectly sensible as a generating function.

Since these ``functions'' we have defined are not really
functions, just formal objects, we should define addition and
multiplication before we proceed.

\begin{defn}

Suppose $f(x)$ and $g(x)$ are two generating functions for
sequences $(a_n)$ and $(b_n)$ respectively.  Then: \be \ii [(i)]
$f(x)+g(x) = \sum_{n=0}^\infty (a_n+b_n)x^n$ \ii [(ii)] $f(x)g(x)
= \sum_{n=0}^\infty (a_nb_0+ a_{n-1}b_1+\cdots + a_0b_n)x^n,$\ee

\end{defn}

Even though this is an abstract definition, these are certainly
the formulas one would obtain if $f$ and $g$ were actual
functions.  To wit, if you add two absolutely convergent power
series together, the coefficients add term-by-term, and if you
multiply two such power series then the coefficient of $x^n$ will
be the sum over $k$ of the coefficient of $x^k$ in the first times
the coefficient of $x^{n-k}$ in the second; such are the ways to
obtain $x^n$ in the product.

Since, however, we are doing everything formally, we should verify
that the rules we are used to working with still hold true here.

\begin{thm}

The following are facts about the arithmetic of generating
functions: \be \ii [(i)] Addition and multiplication are
commutative and associative, \ii [(ii)] There exist additive and
multiplicative identities, and \ii [(iii)] There exist additive
inverses of every function, and multiplicative inverses of
functions exist if and only if the coefficient of $x^0$ is
nonzero. Both types of inverses, when they exist, are unique.

\ii [(iv)] If $f(x)$ and $g(x)$ are generating functions with
$f(x)g(x)=0$, then $f(x)=0$ or $g(x)=0$.

\ee

\end{thm}

\begin{proof}
\be \ii [(i)] Let $f(x)$, $g(x)$, and $h(x)$ be arbitrary
generating functions formed from sequences $(a_n)$, $(b_n)$ and
$(c_n)$ respectively. To check additive commutativity, notice that
\[f(x)+g(x) = \sum_{n=0}^\infty (a_n+b_n)x^n = \sum_{n=0}^\infty
(b_n+a_n)x^n = g(x)+f(x);\] additive associativity is similar. For
multiplicative commutativity, we compute \[ f(x)g(x) =
\sum_{n=0}^\infty \Lp\sum_{k=0}^n a_kb_{n-k}\Rp x^n =
\sum_{n=0}^\infty \Lp\sum_{j=0}^n b_ja_{n-j}\Rp x^n = g(x)f(x).\]
For multiplicative associativity, we compute \[[f(x)g(x)]h(x) =
\sum_{n=0}^\infty\Lp\sum_{j=0}^n\Lp\sum_{k=0}^ja_kb_{j-k}\Rp
c_{n-k}\Rp x^n.\]  The coefficient of $x^n$ is thus the sum over
all nonnegative integers $r,s,t$ with $r+s+t=n$ of $a_rb_sc_t$
(here $r=k$, $s=j-k$, and $t=n-j$. Since we get the same answer if
we rearrange $f$, $g$, and $h$, we see that
\[[f(x)g(x)]h(x)=[g(x)h(x)]f(x).\]  Now by multiplicative
commutativity, we get associativity.

\ii [(ii)] The additive and multiplicative identities are the
functions $0 = 0+0x+0x^2+\cdots$ and $1=1+0x+0x^2+\cdots$,
respectively. It follows directly from the definitions that these
are identities.  In general, if $c$ is a complex number, we will
abuse notation and denote by $c$ the function $c+0x+0x^2+\cdots$;
such functions will be known as \emph{constant functions}.  The
zeroth coefficient of a generating function will also be known as
the \emph{constant term}.  Notice that constant terms add and
multiply within power series as if they were just numbers.

\ii [(iii)] The additive inverse of $f(x) =
a_0+a_1x+a_2x^2+\cdots$ is just $-a_0-a_1x-a_2x^2-\cdots$, as is
easy to verify; it is also easy to verify that this is unique.
Now suppose that a generating function
$f(x)=a_0+a_1x+a_2x^2+\cdots$ has a multiplicative inverse
$g(x)=b_0+b_1x+b_2x^2+\cdots$, so that $f(x)g(x)=1.$ Since the
constant term of $f(x)g(x)$ is $a_0b_0$, we must have $a_0b_0=1$,
so that $a_0\neq 0$ is certainly a necessary condition for $f$ to
have a multiplicative inverse. Now, suppose $a_0\neq 0$.  We know
that if a $g(x)$ as above exists, then $b_0$ must equal $1/{a_0}$.
Now, suppose we have defined all the $b_i$ for $i<n$.  By matching
coefficients of $x^n$, we see that \[a_0b_n + \sum_{k=1}^n
a_kb_{n-k} = 0,\] so that $b_n$ is forced to be
\[-\frac{1}{a_0}\sum_{k=1}^n a_kb_{n-k}.\]  Therefore, only one
$g(x)$ can be the multiplicative inverse of $f(x)$, and this
$g(x)$ works by construction.

\ii [(iv)] Suppose $f(x)$ and $g(x)$ are generating functions
whose product is $0$.  Suppose $g(x)\neq 0$, say with first
nonzero coefficient $b_nx^n$.  Let $f(x)=a_0+a_1x+a_2x^2+\cdots.$
We will show that all the $a_i$ are zero, and thus that $f=0$.
Indeed, for a contradiction let $k$ be least with $a_k\neq 0$.
Consider the coefficient of $x^{n+k}$ in $f(x)g(x)$.  Since $g$
has no coefficients on powers less than $n$ and likewise for $f$
and $k$, this coefficient must be $a_kb_n\neq 0$.  But this
contradicts that $f(x)g(x)=0$, so we're done.

\ee

\end{proof}

Let us now work through a few examples.\\

{\bf Example 1: Geometric Sequences.}  As we saw above, we would
like to say that the geometric sequence $a_n = a_0r^n$ for some
number $r$ gives rise to the generating function
$f(x)=a_0/(1-rx)$.  What could this mean, in the formal world we
are assembling?  Well, it would mean that $f(x)(1-rx) = a_0$,
\emph{as power series}.  This is easy to verify; $a_0\cdot 1 =
a_0$, and for $n\geq 1$ the coefficient of $x^n$ in the product is
just $a_0r^n - (a_0r^{n-1})\cdot r = 0$.  Therefore (if $a_0\neq
0$) the multiplicative inverse of $f(x)$ is $(1-rx)/a_0$, and we
may as well think of $f(x)$ as the ``rational function''
$a_0/(1-rx)$.\\

{\bf Example 2: The Fibonacci Sequence.} Recall the Fibonacci
sequence, defined as $F_0=1,$ $F_1=1$, and $F_{n+1} = F_n+F_{n-1}$
for $n\geq 1$.  Form the generating function \[F(x) = F_0 +
F_1x+F_2x^2+\cdots =1+x+2x^2+3x^3+5x^4+\cdots.\]  Consider the
expression $F(x)(1-x-x^2)$.  The first two terms of this product
are $1+0x+\cdots$.  For the rest, notice that the coefficient of
$x^n$ for $n\geq 2$ is $F_{n}-F_{n-1}-F_{n-2} = 0$, and we find
that \[F(x) = \frac{1}{1-x-x^2}.\]  Here is where generating
functions start to become quite useful.  Let us split up this
rational function in partial fractions.  Let $\varphi =
\frac{1+\sqrt{5}}{2}$ and $\bar{\varphi}=\frac{1-\sqrt{5}}{2}$, so
that $1-x-x^2 = (1-\varphi x)(1-\bar{\varphi}x).$  An easy
computation finds that \[\frac{x}{1-x-x^2} =
\frac{1}{\sqrt{5}}\Lp\frac{\varphi}{1-\varphi
x}-\frac{\bar{\varphi}}{1-\bar{\varphi}x}\Rp.\]  Using the facts
about geometric sequences from the previous example, we conclude
that
\[F(x) =
\sum_{n=0}^\infty\Lp\frac{\varphi^{n+1}-\bar{\varphi}^{n+1}}{\sqrt{5}}\Rp
x^n,\] so that \[F_n =
\frac{\Lp\frac{1+\sqrt{5}}{2}\Rp^{n+1}-\Lp\frac{1-\sqrt{5}}{2}\Rp^{n+1}}{\sqrt{5}}.\]
Thus, we get a closed-form formula for the Fibonacci numbers.\\

Now we introduce a class of generating functions which can prove
quite useful. Before delving in, notice that we can write the
binomial coefficient \[\binom{m}{n} =
\frac{m(m-1)(m-2)(\cdots)(m-n+1)}{n!}.\]  Since this is a
degree-$n$ polynomial in $m$, we may actually make sense of the
expression $\binom{x}{n}$ for any (even complex!) number $x$ (as
long as $n$ remains a nonnegative integer).

\begin{defn}
Let $c$ be a complex number.  The generating function $(1+x)^c$ is
defined to be \[(1+x)^c = \sum_{n=0}^\infty\binom{c}{n}x^n =
1+cx+\binom{c}{2}x^2+\cdots.\]
\end{defn}

We see immediately that $(1+x)^c$ agrees with our usual
interpretation of that expression when $c$ is a nonnegative
integer.  The following theorem will imply (amongst other things)
that this definition is compatible with the notion of $(1+x)^{-1}$
we were using above.

\begin{thm}
Let $c$ and $d$ be complex numbers.  Then the following equality
of generating functions holds: \[(1+x)^c(1+x)^d = (1+x)^{c+d}.\]
\end{thm}

\begin{proof}
Notice that both the left-hand side and the right-hand side of the
desired expression have polynomials in $c$ and $d$ as coefficients
of $x^n$; call these polynomials $p_n(c,d)$ and $q_n(c,d)$.  We
need only show that $p_n=q_n$.  Suppose $c$ and $d$ are
nonnegative integers. The expression $(1+x)^c(1+x)^d =
(1+x)^{c+d}$ holds by the remark above, and so $p_n(c,d)=q_n(c,d)$
for any nonnegative integers $c$ and $d$.  Now fix $c$ (still a
nonnegative integer).  The expression $p_n(c,d)=q_n(c,d)$ holds
for infinitely many $d$ (it holds for all nonnegative integers),
and thus the two polynomials $p_n(c,y)$ and $q_n(c,y)$ must be
identical for every nonnegative integer $c$.  Now let $p_{nm}(c)$
and $q_{nm}(c)$ be the coefficients of $y^m$ in the above
polynomials; these agree for infinitely many $c$, and thus must be
identical polynomials as well.  We conclude that $p_n$ and $q_n$
must have been the same polynomials in the first place, and thus
that $(1+x)^c(1+x)^d=(1+x)^{c+d}$ for any complex $c$ and $d$.
\end{proof}

Therefore, we can raise $1+x$ to negative powers, and also extract
roots.  Notice that in addition to this, we may understand
$(1+rx)^c$ as having the coefficient of $x^n$ in $(1+x)^c$
multiplied by $r^n$; the same argument gives the same result.
Therefore, if $(1-x)^{-1}$ is understood as in the definition
above, we find that $(1-x)^{-1}(1-x) = 1$, and so the two notions
we now have of $(1-x)^{-1}$ coincide (by uniqueness of
inverses).\\

{\bf Example 3: Polynomial Sequences.} Suppose that $f(x)$ is the
generating function $(1-x)^{-m-1}$, where $m$ is an integer.  We
can compute that the coefficient of $x^n$ in this generating
function is
\[(-1)^n\binom{-m-1}{n}=(-1)^n\frac{(-m-1)(-m-2)(-m-3)
\cdots(-m-n)}{n!}=\binom{n+m}{m}.\] Therefore, the sequence of
coefficients is a sequence of degree-$m$ polynomials in $n$,
namely $\binom{n+m}{m}$.  From these we can obtain all
polynomials, so these can be used as a basic set of widgets for
finding generating functions of polynomial sequences.

As an example, let us compute the generating function of the
sequence $a_n=n^2-3n+1.$  We first write the given polynomial in
terms of binomial coefficients: \[n^2-3n+1 = 2\binom{n+2}{2} -
6\binom{n+1}{1}+5\binom{n}{0}.\]  Thus, the generating function
corresponding to this sequence is
\[\frac{2}{(1-x)^3}-\frac{6}{(1-x)^2}+\frac{5}{1-x}.\]  The
generating function of every polynomial can be found in this way,
and conversely, every rational function whose denominator is a
power of $1-x$ and whose numerator is of lesser degree (so that it
breaks up into fractions alone) is the generating function of a
polynomial.

Here again, if we replace $1-x$ with $1-rx$, we get the generating
functions of sequences of the form $a_n=p(n)r^n$ where $p(n)$ is a
polynomial in $n$.  But now we have a method: given a generating
function which is a rational function, we can put it into its
partial fraction decomposition and see that the sequence it
corresponds to is (after a finite amount of time) a sum of
polynomials multiplied by geometric series.  Furthermore, if the
rational function is in lowest terms, then the power of $1-rx$ in
the denominator is one more than the
degree of the polynomial multiplied by $r^n$ in the sequence.\\

{\bf Example 4: Linear Recurrences.}  Suppose we have a linear
recurrence, that is a recurrence of the form $a_{n+k} =
b_1a_{n+k-1}+b_2a_{n+k-2}+\cdots+b_ka_n,$ where the $b_i$ are
complex numbers and such that we are given (finitely many) initial
values (which include at least) $a_0,\ldots,a_{k-1}.$ The
Fibonacci sequence is one such.  Let $f(x)$ be the generating
function of the sequence described above. If we look at the
expression
\[f(x)(1-b_1x-b_2x^2-\cdots-b_kx^k),\] we notice that for
sufficiently large $n$ (say, as soon as we clear the initial
conditions) the coefficient of $x^n$ is $0$, and so the expression
is actually a polynomial $p(x)$.  We conclude that \[f(x) =
\frac{p(x)}{1-b_1x-b_2x^2-\cdots-b_kx^k},\] and so $f(x)$ is a
rational function, so that the discussion at the end of Example 3
applies.  We conclude that we can ``read off'' the form of a
linear recurrence by factoring the polynomial
$1-b_1x-b_2x^2-\cdots-b_kx^k$; this is often easier than actually
finding $p(x)$ and computing the partial fraction decomposition of
$f(x)$.\\

{\bf Example 5: Catalan Numbers.}  Recall the recurrence for the
Catalan numbers $(C_n)$:
\[C_n=C_{n-1}C_0+C_{n-2}C_1+\cdots+C_0C_{n-1}.\]  If we let $C(x)$
be the generating function of the Catalan numbers, then this
relation gives that \[C(x) = xC(x)^2+1.\]  Applying the quadratic
formula, we obtain that \[C(x) = \frac{1\pm\sqrt{1-4x}}{x}.\]
Since the $x$ in the denominator has to cancel an $x$ in the
numerator, the constant term of the numerator must be $0$, and we
see that the sign must be minus.  Therefore,
\[C(x)=\frac{1-\sqrt{1-4x}}{2x}.\]  We thus obtain the formula
\[C_n = 2(-4)^n\binom{1/2}{n+1},\] whose coincidence with the
usual formula is not hard to show.\\

{\bf Example 6: Partitions.} Let $\pi(n)$ denote the number of
unordered partitions of $n$, so that e.g. $\pi(5)= 7$ because
$5=4+1=3+2=3+1+1=2+2+1=2+1+1+1=1+1+1+1+1$ is seven partitions of
5.  This is in general a very difficult sequence to work with, but
it does have a relatively well-behaved generating function.  Let
$\Pi(x)$ be the generating function corresponding to the partition
sequence.  Let us look at a partition of $n$ as a factorization of
$x^n$, so that the partition $5=3+1+1$ above corresponds to the
factorization $x^5=x^3\cdot (x^1)^2.$  We thus would like a
contribution of exactly 1 to the coefficient of $x^n$ for each
sequence $k_1,\ldots,k_r$ with
\[(x^1)^{k_1}(x^2)^{k_2}\cdots(x^r)^{k_r}=x^n.\]  But this is
accomplished by the following infinite product:
\[\prod_{k=1}^\infty(1+x^k+x^{2k}+x^{3k}+\cdots)=
\prod_{k=1}^\infty\frac{1}{1-x^k}.\]

Beyond addition and multiplication, there are other operations
that we may define on generating functions, namely composition and
exponentiation, as well as differentiation.  These, however, are a
topic for another handout.
\newpage
\begin{center}
{\bf Exercises and Problems}
\end{center}
\be

\ii [1.] Compute the coefficient of $x^{21}$ in the expansion of
\[
(x+x^2+x^3+x^4+x^5+x^6)^6.
\]

\ii [2.] Let $n$ be a positive integer, and let
\[
f(x) = \sum_{k=0}^{n}\binom{n}{k}^2 (1+x)^{2n-2k}(1-x)^{2k}.
\]
Show that the coefficient of $x^{2m-1}$ in $f(x)$ is 0 for all
positive integers $m$.

\ii [3.] Express
\[
\binom{n}{0}^2 - \binom{n}{1}^2 + \binom{n}{2}^2 - \cdots +
(-1)^n\binom{n}{n}^2
\]
in closed form.

\ii [4.] (Cauchy's Identity) Let $n$ be a positive integer. Prove
that
\[
\sum_{\stackrel{\scriptstyle k_1 + 2k_2 + \cdots + nk_n = n}{k_1,
k_2, \dots , k_n \ge 0}} \frac{1}{k_1!k_2!\cdots k_r!1^{k_1}
2^{k_2}\cdots n^{k_n}} = 1,
\]

\ii [5.] Let $\ell$ be an even positive integer. Express
\[
\sum_{k=0}^{n}\sum_{i=0}^{\ell} (-1)^i
\binom{n}{k}^2\binom{2k}{i}\binom{2n-2k}{\ell-i}
\]
in closed form.

\ii [6.] (Euler's Pentagonal Numbers Theorem) Let $\Pi(x)$ be the
generating function of the partition sequence, and let $f(x)$ be
the generating function defined as \[f(x) = 1+\sum_{k=1}^\infty
(-1)^k\Lp x^{k(3k-1)/2}+x^{k(3k+1)/2}\Rp.\] Prove that $\Pi(x)f(x)
= 1.$

\ee

\end{document}
