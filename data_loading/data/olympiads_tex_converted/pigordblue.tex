\documentclass[11pt]{article}
\usepackage{amsfonts, latexsym, epsf}
\usepackage{amsmath}

%\setcounter{page}{15}
\setcounter{section}{0} \pagestyle{myheadings}
\def\leftmark{\scriptsize\it{MOSP 2004, Blue Group, Pigeonhole Principle and Well-ordering\hfill Zuming Feng,
Phillips Exeter Academy, zfeng@exeter.edu}}
\def\rightmark{\scriptsize\it{MOSP 2004, Blue Group, Pigeonhole Principle and Well-ordering\hfill Zuming Feng,
Phillips Exeter Academy, zfeng@exeter.edu}} \makeatletter
\def\@evenhead{
\vbox{\hbox to\hsize{\bf \thepage \hfill \sl \leftmark}
\vspace{2pt} \hbox to\hsize{\hrulefill}}}
\def\@oddhead{
\vbox{\hbox to\hsize{\bf \rightmark \hfill \bf \thepage}
\vspace{2pt} \hbox to\hsize{\hrulefill}}}
\makeatother


\def\textbf#1{{\bf #1}}
\def\textit#1{{\it #1}}
\def\mathrm#1{{\mbox #1}}
\def\newline{\linebreak}

\setlength{\textwidth}{7in}
\setlength{\oddsidemargin}{-0.35in}
\setlength{\evensidemargin}{-0.35in}
\setlength{\textheight}{9in}
\setlength{\topmargin}{-0.35in}
\setlength{\baselineskip}{2pt}

\def\OR{\begin{center} \textbf{ OR } \\ \end{center}}
\def\soln{{\bf \noindent \newline Solution:  } \nopagebreak}
\def\first{{\bf \noindent \newline First Solution:  } \nopagebreak}
\def\second{{\bf \noindent \newline Second Solution:  } \nopagebreak}
\def\third{{\bf \noindent \newline Third Solution:  } \nopagebreak}
\def\fourth{{\bf \noindent \newline Fourth Solution:  } \nopagebreak}
\def\fifth{{\bf \noindent \newline Fifth Solution:  } \nopagebreak}
\def\sixth{{\bf \noindent \newline Sixth Solution:  } \nopagebreak}
\def\note{{\bf \noindent \newline Note:  } \nopagebreak}
\def\comment{{\bf \noindent \newline Comment:  } \nopagebreak}

\def\be{\begin{enumerate}}
\def\ii{\item}
\def\ee{\end{enumerate}}
\def\bi{\begin{itemize}}
\def\ei{\end{itemize}}

\def\CC{\mathbb{C}}
\def\mathbb{N}{\mathbb{N}}
\def\QQ{\mathbb{Q}}
\def\RR{\mathbb{R}}
\def\mathbb{Z}{\mathbb{Z}}

\def\calA{\mathcal{A}}
\def\calB{\mathcal{B}}
\def\calC{\mathcal{C}}
\def\calF{\mathcal{F}}
\def\calG{\mathcal{G}}
\def\calP{\mathcal{P}}
\def\calR{\mathcal{R}}
\def\calS{\mathcal{S}}
\def\calT{\mathcal{T}}

\def\lf{\lfloor}
\def\rf{\rfloor}
\def\floor#1{\lf {#1} \rf}
\def\Lf{\left\lfloor}%tjp
\def\Rf{\right\rfloor}%tjp
\def\lc{\lceil}
\def\rc{\rceil}
\def\Lp{\left(}
\def\Rp{\right)}
\def\Lb{\left[} %tjp
\def\Rb{\right]}%tjp
\def\Paren#1{\Lp {#1} \Rp}

\def\ang{\angle}
\def\sang{\sin \angle}
\def\oar{\overrightarrow}
\def\ray{\stackrel\rightharpoonup}
\def\odar{\stackrel\longleftrightarrow}
\def\ol{\overline}
\def\sfn{\widehat}

\def\beqa{\begin{eqnarray*}}
\def\eeqa{\end{eqnarray*}}
\def\dsp{\displaystyle}

\def\binom#1#2{{#1 \choose #2}}
\def\half{\frac{1}{2}}
\def\lcm{\textrm{lcm}}
\def\dnd{\mbox{\ $\!\not\;\mid $\ }}
\def\arc{\widehat}
\def\cis{\,\mbox{cis}\,}
\def\real{\,\mbox{Re}\,}
\def\legendre#1#2{\left( \frac{#1}{#2} \right)}
\def\dg{^{\circ}}
\def\mymod#1{\,(\mbox{mod}\,\, #1)}
\def\map#1#2#3{#1\!: #2\to #3}
\def\eop{\rule{5pt}{5pt}}

\def\blankpage{\newpage
\thispagestyle{empty}
\begin{center}
%Blank Page
\end{center}
\newpage}

\newcounter{newtheorem}
\newtheorem{lemma}{Lemma}
%\newtheorem*{lemma*}{Lemma}

%\look Are these correct
\def\cycsum{\sum_{\textrm{\footnotesize cyc}}}
\def\symsum{\sum_{\textrm{\footnotesize sym}}}

%\look ... Reid's macros
\def\osqo#1{\multicolumn{1}{c}{#1}}
\def\vsqo#1{\multicolumn{1}{|c}{#1}}
\def\osqv#1{\multicolumn{1}{c|}{#1}}
\def\vsqv#1{\multicolumn{1}{|c|}{#1}}
\def\yy{\hbox to 4pt{\vbox to 8pt{\vfil}\hfil}}

\def\stacksubscript#1#2{_{\scriptstyle #1\atop \scriptstyle #2}}

\def\script{\scriptsize}
% \def\th{^{\script\mbox{th}}}
% \def\st{^{\script\mbox{st}}}
% \def\nd{^{\script\mbox{nd}}}

\def^{\text th}{^{\mbox{\script th}}}
\def\st{^{\mbox{\script st}}}
\def\nd{^{\mbox{\script nd}}}

\def\rad{\textrm{rad}}

%\newcounter{felist}
%\def\bfe{\begin{list}{\alph{felist}}{\usecounter{fe}
\def\bi{\begin{itemize}}
\def\ei{\end{itemize}}
\def\dd{\dots}
\def\half{\frac{1}{2}}
\def\sang{\sin \angle}

\def\msk{\medskip\noindent}
\def\bsk{\bigskip\noindent}

\def\bou{\mathbf{u}}
\def\bov{\mathbf{v}}
\def\bow{\mathbf{w}}


\begin{document}
\newpage

\begin{flushright}
{\bf MOSP Blue Group\\
June 25, 2004}
\end{flushright}



\begin{center}
{\Large\bf Pigeonhole Principle and Well-ordering}
\end{center}

\be
\ii
Twenty five boys and twenty five girls sit around a table. Prove
that, no matter what the arrangement, it is always possible to
find someone sitting in between $2$ girls.

%Wang Lianxiao p41/7.
\ii
%[9.7.]
%[IMO 1968]
Prove that in every tetrahedron there is a vertex such that the
three edges meeting there have lengths which are the sides of a
triangle.

\ii     %{[{31.}]}     %%@@@@@@@@@@@@@  ANOTHER PROBLEM!  @@@@@@@@@@@@@%%
%proofread 3/16/02 --pll
%{[China 1990]}
In an arena, each row has 199 seats. One day, 1990 students are
coming to attend a soccer match. It is only known that at most 39
students are from the same school. If students from the same
school must sit in the same row, determine the minimum number of
rows that must be reserved for these students.

\ii
%[8.1.] {[Poland 98]}
For $ i = 1, 2, \cdots , 7$ $a_i$ and $b_i$ are nonnegative
numbers such that $a_i + b_i \le 2.$ Prove that there are 2
distinct indices $k, m \in \{ 1, 2, \cdots , 7 \}$ such that $|a_k
- a_m| + |b_k - b_m| \le 1.$

%Wang Lianxiao p41/6.
\ii
%[9.11.]
Prove that, for any convex pentagon, three of its diagonals can
form sides of a triangle.

\ii
%[9.12.] [China 1963]
There are 99 points in a plane.
\be
\ii [(a)]
Assume that no three are collinear and no four lie on a circle.
Prove that it is always possible to choose three of the points and
draw a circle through these points, such that exactly 48 of the
remaining points lie inside this circle.
\ii [(b)]
Prove that it is always possible to draw a circle such that it
passes exactly 1 of the given points, encloses exactly 49 of the
given points.
\ee

\ii
%[8.2.] [St. Petersburg 89]
Ninety-one white pawns are placed on a $10\times 10$ chessboard.
Judy repaints these pawns black one at a time and puts down each
repainted pawn on any empty square of the board. Judy tries to
avoid the situation of two pawns of different colors occupying two
squares that have a common side. Can she do it?

\ii
Let $\calP$ be a 13-side polygon. (Note that $\calP$ is not
necessarily convex.) Prove that there is a line $\ell$ such that
there is exactly one side of $\calP$ lie on $\ell$.

\ii%Easy-Med Kvant 303
Given a $300 \times 1000$ unit square grid, thirty equal weights
are placed on some of the vertices of the grid (where one vertex
has at most one weight). Prove that there exist two
non-overlapping groups of no more than 10 weights each having the
same center of mass.

\ii
Let $A = \ol{a_1a_2\cdots a_{16}}$ be a 16-digit number. Prove
that it is possible to find integer(s) $1\le i \le j \le 16$ such
that the product $a_i\cdots a_j$ is a perfect square.


\ii
A chess player has been trained for $11$ weeks $(77$ days). During
the period, he played at least $1$ game each day and at most $12$
games each week. Prove that there  exist some consecutive dates
such that he played exactly $21$ games during the period.

\ii
Prove that in an infinite sequence $\{ a_k\}$ of integers,
pairwise distinct and each term greater than 1, one can find a
subsequence $\{ a_{i_k} \}$  such that $a_{i_k} > i_k$ for all
positive integers $k$.

\ii
Given positive integers $a_1, a_2, \cdots a_{m+1}$ with $a_1 + a_2
+ \cdots + a_{m+1} = 2m$ and $a_i\le m$ for all $i = 1, 2, \cdots
, m+1,$ prove that it is always  possible to separate the numbers
into two groups, each group having its total equals to $m.$

\ii
There are some circles inside a square with side $1.$ Suppose that
the total circumferences of circles is $10.$ Prove that there is a
line which can cut through $4$ of those circles.

\ii
%[9.15.] [Moscow 1995]
%Zhongshu
A system of right triangles is constructed as following. The
system begins with 4 congruent right triangles. One at a time, one
triangle in the system is picked and cut into two triangles along
the altitude to the hypotenuse of the original triangle. Prove
that, at any time, there are at least two congruent triangles.

\ii
There are $4$ multiple choice problems on a certain test.
Each problem has 3 solutions to be chosen from. A group of
students take the test to generate the following  interesting
situation: for every three of them, there is a problem such that
they all have different answers. Determine, with justification,
the maximum number of students taking the test.


\ii
Prove that, for any odd prime $p,$ there exists integers $x$ and
$y$ such that $p|(1 + x^2 + y^2).$

\ii
Given positive distinct integers $a_1, a_2, \cdots , a_{101}$ all
less than 5050, prove that there exist 4 different numbers $a_k,
a_l, a_m, a_n$ such that the number $a_k +  a_l - a_m - a_n$ is
divisible by 5050.

\ii
%[9.21.] [Paul Zietz]
Twenty-three people, each with integral weight, decide to play
football, separating into two teams of eleven people, plus a
referee. To keep things fair, the teams chosen must have equal
{\em\underline{total}} weight. It turns out that no matter who is
chosen to be the referee, this can always be done. Prove that the
23 people must all have the same weight.

\ii
Prove that among any 16 distinct positive integers not exceeding
100 there are 4 different numbers among them, $a, b, c, d,$ such
that $a + b = c + d.$

\ii
Prove that there exists a positive integer $n$ such that the three
most left digits of the decimal representation of $2^n$ is 999.

\ii
%{[MOSP 2002, from Kvant]}
Consider a finite number of polygons in the plane such that any
two have at least a common point. Prove that there exists a line
that has at least a common point with each of the given polygons.

\ii
%[9.23.] [China 1986]
In a tournament, every two players have exactly one game between
each other. There is no ties. A player $A$ is awarded a prize if
for every other player $B$, either $A$ defeats $B$, or there is a
third player $C$ such that $A$ defeats $C$ and $C$ defeats $B.$
Suppose that only one player is awarded a prize. Prove that this
player defeats every other player.

\ii
%[9.24.] [China 1986]
Let $z_1, z_2, \cdots , z_n$ be complex numbers such that $|z_1| +
|z_2| + \cdots + |z_n| = 1.$ Prove that the sum $s$ of a subset of
$\{ z_1, z_2, \cdots , z_n \}$ satisfies $|s| \ge 1/4.$

\ii
%IMO03 Shortlist
Let $x_1, x_2, \dots , x_n$ and $y_1, y_2, \dots , y_n$ be real
numbers. Let $\mathbf{M} = (a_{ij})_{1\le i, j\le n}$ be the
matrix with entries
\[
a_{ij} = \left\{
\begin{array}{ll}
1, & \mbox{if $x_i+y_j \ge 0;$} \\
0, & \mbox{if $x_i+y_j < 0.$}
\end{array}
\right.
\]
Suppose that $\mathbf{N}$ is an $n\times n$ matrix with entries 0
or 1 in such a way that the sum of the elements in each row and
column of $\mathbf{N}$ is equal to the corresponding sum for the
matrix $\mathbf{M}$. Prove that $\mathbf{M} = \mathbf{N}$.

\ii
%\msk %Kvant 322
%{\bf Example 2.2.6.}\quad
%{\it {[Kvant]}
An alphabet consists of $n$ letters. One writes letters around a
circle such that
\be
\ii [(i)]
no one appears twice in row; and
\ii [(ii)]
for any two distinct letters $a$ and $b$ one can draw a line such
that all letters $a$ lie on one side of the line and all letters
$b$ on the other side.
\ee
Determine the maximum number of letters that can be written.

\ii %[2.2.10.]
%{[MOSP 2001, from Kavant]}
A region is the intersection of exactly $n$ circular discs. Prove
that its boundary consists of at most $2n-2$ circular arcs.

\ii
%MMM
%West CMO02/2/4
Let $S = (a_1, a_2, \dots , a_n)$ be the longest binary sequence
such that for $1 \le i < j \le n-4$,
\[
(a_{i}, a_{i+1}, a_{i+2}, a_{i+3}, a_{i+4}) \ne (a_j, a_{j+1},
a_{j+2}, a_{j+3}, a_{j+4}).
\]
Prove that $(a_1, a_2, a_3, a_4) = (a_{n-3}, a_{n-2}, a_{n-1},
a_{n})$.

\ii
%[China 92]
Let
\[
S = \{ (x, y)\| {\mbox{$x, y$ are integers and $-2 \le x, y \le 2$
}} \}
\]
be a set of points on the plane, and let $P_1, P_2, \dots , P_6$
be any six points in $S$. Prove that there are three distinct
indices $i, j, k$ such that $[P_iP_jP_k] \le 2$.

%Baltic 95
\ii
%[8.10.]
\be
\ii [(a)]
Given a $2n+1$ convex polygon. Determine if it is possible to
assign numbers 1, 2, $\cdots,$ $4n+2$ to the vertices and
midpoints of the sides of the sides of the polygon such that for
each side the sum of the three numbers assigned to it is the same.
%Baltic 94.
\ii [(b)]
Lines $\ell_1,$ $\ell_2,$ $\cdots,$ $\ell_k$ are in general
position in the plane (no two are parallel and no three are
concurrent). For which values of $k$ can we label the intersection
points of these lines by the numbers $1, 2, \cdots , k-1$ so that
for each of the lines $\ell_1,$ $\ell_2,$ $\cdots,$ $\ell_k$ all
the labels appear exactly once?
\ee

%USSR MO p54.
\ii
%[8.11.]
Form a $2003\times 2005$ screen by unit screens. There are at
least $2002\times 2004$ unit screens which are {\em on.} In any
$2\times 2$ screen, as soon as there are $3$ unit screens which
are {\em off,} the $4^{\text th}$ screen will be turned off
automatically. Prove that the whole screen can never be totally
off.

%Baltic 97.
\ii
%[8.12.]
In a forest each of $9$ animals lives in its own cave, and there
is exactly one separate path between any two of these caves.
Before the election for Forest Gump, King of the Forest, some of
the animals make an election campaign. Each campain-making animal
-- ${\mathcal{FGC}}$ (Forest Gump candidates) visits each of the
other caves exactly once, use only the paths for moving from cave
to cave, never turns from one path to another between the caves
and returns to its own cave in the end of campaign. It is also
know that no path between two caves is used by more than one
${\mathcal{FGC}}.$ Find the maximum number of ${\mathcal{FGC}}.$


\ii
An equilateral triangle is divided into 9,000,000 congruent
equilateral triangles by lines parallel to its sides. Each vertex
of the small triangles is colored in one  of three colors. Prove
that there exist three points of the same color being the vertices
of  a triangle with its sides parallel to the lines of the
original triangle.

\ii
Given any $n+2$ distinct numbers from the first $3n$ positive
integers, prove that it is always possible to find two numbers,
say $a$ and $b$ such that $n < a - b < 2n.$

\ii
%MOSP04 rt1.3
Given eight distinct positive integers not exceeding 2004, prove
that there are four of them, say $a, b, c$, and $d$, such that
\[
4 + d \le a + b + c\le 4d.
\]

\ii
%MOSP04 t1
%IMO03 Shortlist
Let $D_1, D_2, \dots$, and $D_n$ be closed discs in the plane.
Suppose that every point in the plane is contained in at most 2003
discs. Prove that there exists a disc which intersects at most
14020 other discs.

(A closed disc is the region bounded by a circle, taken jointly
with this circle.)

\ii
Given 200 integers, prove that it is always possible to find 100
of them such that their sum is divisible by 100.

\ii
%{[China 1998]}
Let $x_{(i, j)},$ $i = 1, 2, \dots ,100,$ $j = 1, 2, \dots , 25,$
denote the entries of a $100 \times 25$ matrix $\mathbf{M}$ For $j
= 1 , 2, \dots 25,$ all the  entries in the $j^{\text th}$ column $x_{(1,
j)},$ $x_{(2, j)},$ $\dots ,$ $x_{(100,j)}$ are to be rearranged
in non-increasing order from top to bottom to obtain the $j^{\text th}$
column $x'_{(1, j)} \ge x'_{(2, j)} \ge
\cdots \ge x'_{(100, j)}$ of the new matrix ${\mathbf{M}}'.$  Suppose
that
\[
\sum_{j=1}^{25} x_{(i, j)} \le 1\quad {\mbox{for all}}\quad
i = 1, 2, \dots , 100.
\]
Determine the minimum value of positive integers $k$ such that,
\[
\sum_{j=1}^{25} x'_{(i, j)} \le 1\quad {\mbox{for all}}\quad
i \ge k.
\]

%\glossary pigeonhole
%\problem[comb]
%China 00, Chengzhang Li
\ii
An exam paper consists of 5 multiple-choice questions, each with 4
different choices; 2000 students take the test, and each student
chooses exactly one answer per question. Find the smallest value
of $n$ for which it is possible for the students' answer sheets to
have the following property: among any $n$ of the students' answer
sheets, there exist 4 of them among which any two have at most 3
common answers.

\ii %[3.4.]
%Russia 00 r5/11.8
Every cell of a $100 \times 100$ board is colored in one of 4
colors so that there are exactly 25 cells of each color in every
column and in every row. Prove that one can choose two columns and
two rows so that the four cells where they intersect are colored
in four different colors.

\ii
The integers from 1 to 100 are written on a lottery ticket. When
one buys a lottery ticket, one chooses 10 of these 100 numbers.
Then 10 of the integers from 1 to 100  are drawn, and a winning
ticket is one which does not contain any of them. Determine the
minimum number of tickets one needs to buy so one can choose
numbers in a way such that it is guaranteed to have at least one
winning ticket.

\ii
Given a square board of size $n \times n$, where $n$ is an integer
greater than 1, we label some of the squares by distinct numbers
from the set $\{1,2,\dots,n^2\}$. What is the largest number of
board squares we can label in this way without creating a
difference of $n$ or higher between any two labels on neighboring
squares on the board?


\ee

\end{document}
