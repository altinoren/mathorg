\documentclass[11pt]{article}
\usepackage{latexsym, amsthm, amsfonts, amsmath}
\usepackage{amssymb}
\setlength{\textwidth}{6.5in} \setlength{\oddsidemargin}{0in}
\setlength{\textheight}{8.5in} \setlength{\topmargin}{0in}
\setlength{\headheight}{0in} \setlength{\headsep}{0in}
\setlength{\parskip}{0pt} \setlength{\parindent}{20pt}

\def\bi{\begin{itemize}}
\def\ei{\end{itemize}}
\def\ii{\item}

\def\be{\begin{enumerate}}
\def\ee{\end{enumerate}}
\def\beqa{\begin{eqnarray}}
\def\eeqa{\end{eqnarray}}
\def\msk{\medskip}

\def\ol{\overline}
\def^{\text th}{^{\mathrm {th}}}
\def\Rc{\rceil}
\def\Lc{\lceil}
\def\Rf{\rfloor}
\def\Lf{\lfloor}
\def\Lp{\left(}
\def\Rp{\right)}
\def\mathbb{N}{\mathbb N}

\begin{document}

\noindent {\bf MOSP 2005, Red}

\noindent {\bf Bijections}

\bigskip

Today, we focus on a key technique in solving combinatorial
problems.

\begin{quote}
  In order to count the elements of a certain set, we replace them with
  those of another set that has the same number of elements and whose
  elements are more easily counted.
\end{quote}

The following theorem is rather intuitive, but we will use it
(implicitly) throughout.

\msk {\bf Theorem 1.}\quad {\it Let $A$ and $B$ be finite sets,
and let $f$ be a function from $A$ to $B$. If $f$ is injective,
then there are at least as many elements in $B$ as in $A$.  If $f$
is surjective, then there are at least as many elements in $A$ as
in $B$.  Thus, if $f$ is bijective, then $A$ and $B$ have the same
number of elements.}

The remainder of today will be consumed with examples of problem
solutions using this approach (as well, sometimes, as others).

\msk {\bf Example 1.}\quad Let $n$ be a positive integer. In how
many ways can one write a sum of (at least two) positive integers
that add up to $n$? Consider the same set of integers written in a
different order as being different. (For example, there are $3$
ways to express 3 as $3 = 1 + 1 + 1 = 2 + 1 = 1 + 2$.)\\

You might think of this as an ``easy'' partition problem; a much
more difficult (and interesting) case is when you consider
unordered partitions.

\msk {\bf Example 2.}\quad {[AHSME 1992]} Ten points are selected
on the positive $x$-axis, $\bf{X}^+$, and five points are selected
on the positive $y$-axis, $\bf{Y}^+$. The fifty segments
connecting the ten points on $\bf{X}^+$ to the five points on
$\bf{Y}^+$ are drawn. What is the maximum possible number of
points of intersection of these fifty segments in the interior of
the first quadrant?

\msk
{\bf Example 3.}\quad %
{[China 1991, by Weichao Wu]} Let $n$ be an integer with $n\ge 2$,
and define the sequence $S = (1, 2, \dots , n)$. A subsequence of
$S$ is called arithmetic if it has at least two terms and it is an
arithmetic progression. An arithmetic subsequence is called
maximal if this progression cannot be lengthened by the inclusion
of another element of $S$ (i.e. one cannot add additional elements
of $S$ on either side and preserve that the sequence is
arithmetic). Determine the number of maximal arithmetic
subsequences.

I believe you have seen the following before, but let us recall
the proof:

\msk {\bf Theorem 2.}\quad {\it Let $m$ and $n$ be positive
integers. \be \ii [(a)] There are $\binom{n-1}{m-1}$ ordered
$m$-tuples $(x_1, x_2, \dots , x_m)$ of positive integers
satisfying the equation $x_1 + x_2 + \dots + x_m = n$. \ii [(b)]
There are $\binom{n+m-1}{m-1}$ ordered $m$-tuples $(x_1, x_2,
\dots , x_m)$ of nonnegative integers satisfying the equation $x_1
+ x_2 + \dots + x_m = n$. \ee}

{\it Proof.}  Routine.

\msk {\bf Example 4.}\quad Setec Astronomy, Inc. has 15 senior
executive officers. Each officer has an access card to the
company's central vault. There are $m$ distinct codes stored in
the magnetic strip of each access card. To open the vault, each
officer who is present puts his access card in the vault's
electronic lock. The computer system then collects all of the
distinct codes on the cards, and the vault is unlocked if and only
if the set of the codes matches the set of $n$ (distinct)
preassigned codes. For security reasons (the company has too many
secrets), the vault can be opened if and only if at least six of
the senior officers are present. Find the values of $n$ and $m$
such that $n$ is minimal and the vault's security policy can be
achieved. (The elements in a set have no order.)

\msk
{\bf Example 5.}\quad %Xin ting ding557
A triangular grid is obtained by tiling an equilateral triangle of
side length $n$ by $n^2$ equilateral triangles of side length 1
(Figure 4.6). Determine the number of parallelograms bounded by
line segments of the grid.

\msk {\bf Example 6.}\quad Bart works at the Lincoln
SuperMegaCineplex 1, which has only one screening room; this one
room seats 200. On the opening night of Star Wars Episode VII: The
Cloned Sith Menace, 200 people are standing in line to buy tickets
for the movie. The cost of each ticket is \$5. Among the 200
people buying tickets, 100 of them have a single \$5 bill, and 100
of them have a single \$10 bill. Bart, being careless, finds
himself with no change at all. The 200 people are in random order
in line, and no one is willing to wait for change when they buy
their ticket. What is the probability that Bart will be able to
sell all of the tickets successfully?

\msk {\bf Example 7.}\quad Let $n$ be a positive integer
satisfying the following property: If $n$ dominoes are placed on a
$6\times 6$ chessboard with each domino covering exactly two unit
squares, then one can always place one more domino on the board
without moving any other dominoes. Determine the maximum value of
$n$.

\msk {\bf Example 8.}\quad {[China 2000, by Jiangang Yao]} Let $n$
be a positive integer and define
\[
M = \{ (x, y) \mid x, y \in \mathbb{N}, 1 \le x, y \le n \}.
\]
Determine the number of functions $f$ defined on $M$ such that \be
\ii [(i)] $f(x, y)$ is a nonnegative integer for any $(x, y) \in
M$; \ii [(ii)] for $1 \le x \le n$, $\sum_{y=1}^{n} f(x, y) =
n-1$; \ii [(iii)] if $f(x_1, y_1)f(x_2, y_2) > 0$, then
$(x_1-x_2)(y_1-y_2) \ge 0$. \ee

\msk {\bf Example 9.}\quad Let $n$ be a positive integer, and let
$A$ denote the set of all increasing partitions of $n$. Let $a =
(a_1, a_2, \dots , a_m)$ be an element of $A$. Let $s(a)$ denote
the smallest index such that $a_{s(a)}, a_{s(a)+1}, \dots , a_m$
are consecutive integers; that is, partition $a$ ends with
$m-s(a)+1$ consecutive integers. Further, assume that $n$ cannot
be written in the form $\frac{k(3k-1)}{2}$ or $\frac{k(3k+1)}{2}$
for any positive integer $k$. Let $A_1$ be a subset of $A$ such
that $a \in A_1$ if and only if $a_1 \le m-s(a)+1$. Show that $|A|
= 2|A_1|$.

\pagebreak

\begin{center}
{\bf Exercises and Problems}
\end{center}

\bigskip


\be \ii [1.] {[AIME 2001]} A fair die is rolled four times. What
is the probability that each of the final three rolls is at least
as large as the roll preceding it?

\ii [2.] {[St. Petersburg 1989]} Tram tickets have six-digit
numbers (from $000000$ to $999999$). A ticket is called {\em
lucky} if the sum of its first three digits is equal to that of
its last three digits. A ticket is called {\em medium} if the sum
of all of its digits is 27. Let $A$ and $B$ denote the numbers of
lucky tickets and medium tickets, respectively. Find $A - B$.

\ii [3.] Let $n$ be a positive integer. Points $A_1, A_2, \dots ,
A_n$ lie on a circle. For $1\le i < j \le n$, we construct
$\ol{A_iA_j}$. Let $S$ denote the set of all such segments.
Determine the maximum number of intersection points can produced
by the elements in $S$.

\ii [4.] %10
Let $A = \{ a_1, a_2, \dots , a_{100} \}$ and $B = \{ 1, 2, \dots
, 50\}$. Determine the number of surjective functions $f$ from $A$
to $B$ such that $f(a_1) \le f(a_2) \le \dots \le f(a_{100})$.
What if $f$ does not need to be surjective?

\ii [5.] {[AIME 1983]} For $\{ 1, 2, \dots , n\}$ and each of its
nonempty subsets a unique {\em alternating sum} is defined as
follows: Arrange the numbers in the subset in decreasing order and
then, beginning with the largest, alternately add and subtract
successive numbers. (For example, the alternating sum for $\{ 1,
2, 4, 6, 9\}$ is $9-6+4-2+1 = 6$ and for $\{ 5\}$, it is simply
5.) Find the sum of all such alternating sums for $n = 7$.

\ii [6.] For a positive integers $n$ denote by $D(n)$ the number
of pairs of different adjacent digits in the binary representation
of $n.$ For example, $D(3) = D(11_2)=0$ and $D(21) = D(10101_2) =
4$. For how many positive integers $n$ less than or equal to 2003
is $D(n)=2?$

\ii [7.] Let
\[
\prod_{n=1}^{1996}\left(1+nx^{3^{n}} \right) = 1 + a_{1}x^{k_{1}}
+ a_{2}x^{k_{2}} + \dots + a_{m}x^{k_{m}},
\]
where $a_{1}, a_{2}, \dots, a_{m}$ are nonzero and $k_{1} < k_{2}
< \dots < k_{m}$. Find $a_{1234}$.

\ii [8.] {[Putnam 2002]} Let $n$ be an integer greater than one,
and let $T_n$ be the number of nonempty subsets $S$ of $\{1, 2, 3,
\dots, n\}$ with the property that the average of the elements of
$S$ is an integer. Prove that $T_n - n$ is always even.

\ii [9.] Let $p(n)$ denote the number of partitions of $n$, and
let $p(n, m)$ denote the number of partitions of $n$ of length
$m$. Prove that $p(n) = p(2n, n)$.

\ii [10.]
%hua2ti 271
A triangular grid is obtained by tiling an equilateral triangle of
side length $n$ with $n^2$ equilateral triangles of side length 1.
Determine the number of rhombuses of side length 1 bounded by line
segments of the grid.

\ii [11.] Suppose that $P_1P_2\dots P_{325}$ is a regular
325-sided polygon. Determine the number of incongruent triangles
of the form $P_iP_jP_k$, where $i, j$, and $k$ are distinct
integers between 1 and 325, inclusive.

\ii [12.] %1 %KED2
{[USAMO 1996, by Kiran Kedlaya]} An ordered $n$-tuple
\[
(x_1, x_2, \ldots, x_n)
\]
in which each term is either 0 or 1 is called a {\em binary
sequence of length} $n$. Let $a_n$ be the number of binary
sequences of length $n$ containing no three consecutive terms
equal to 0, 1, 0 in that order.  Let $b_n$ be the number of binary
sequences of length $n$ that contain no four consecutive terms
equal to 0, 0, 1, 1 or 1, 1, 0, 0 in that order. Prove that
$b_{n+1} = 2a_n$ for all positive integers $n$.

\ee

\end{document}
